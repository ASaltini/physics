\chapter{Newtonian Mechanics}

\section{Constraints}

In Newtonian Mechanics, constraints manifest in terms of forces that prevent the
system from violating the constraint.
These \textbf{constraint forces} are usually denoted with \(\vv{\Phi}\).
The value of such forces depends not only upon the constraint function, but also
on other forces present in the system.
Consider, as an example, a book lying on a table: the amount of force required
to prevent the book from falling through the table depends not only on the shape
of the table, but also on the weight of the book and other forces that might be
pushing the book towards the table.

This gives us a new way to categorise constraints, based on their reaction forces.
Here we denote with \(\vv{\Phi}_s\) the force acting on the \(s\)-th particle,
the coordinates of which are given by \(\vv{r}_s\).
%
\begin{itemize}
  \item \textbf{Smooth:} there are no components of the reaction force along the
  unconstrained directions. Mathematically, this means that the constraint force
  is in the same direction as the gradient of the constraint function:
  \[\cpd{\vv{\Phi}_s}{\pderiv{f}{\vv{r}_s}} = 0\]
  \item \textbf{Rough:} the reaction force may have components along the unconstrained
  directions. This is usually the case in systems where dry friction is present.
  \[\cpd{\vv{\Phi}_s}{\pderiv{f}{\vv{r}_s}} \neq 0\]
\end{itemize}
