\chapter{Newtonian Mechanics}

\section{Constraints}

A \textbf{constraint} is a relation between coordinates and velocities, reducing
the number of degrees of freedom of a system.
Using a set \(\vv{q}\) of generalised coordinates, it can be represented as an
equality or inequality involving a \textbf{constraint function} \(f(\vv{q},\vd{q},t)\).

In Newtonian Mechanics, constraints manifest in terms of forces that prevent the
system from violating the constraint.
These \textbf{constraint forces} are usually denoted with \(\vv{\Phi}\).
