\chapter{General concepts}

\section{Constraints}

A \textbf{constraint} is a relation between coordinates and velocities, reducing
the number of degrees of freedom of a system.
Using a set \(\vv{q}\) of generalised coordinates, it can be represented as an
equality or inequality involving a \textbf{constraint function} \(f(\vv{q},\vd{q},t)\).

There is several ways to classify constraints.
\begin{itemize}
  \item Directionality:
  \begin{itemize}
    \item \textbf{Unilateral:} the constraint may prevent displacements in one direction, but
    not necessarily in the opposite direction. These constraints generally represent
    bodies that cannot be penetrated. Mathematically, they are represented by
    inequalities: \(f(\vv{q},\vd{q},t) \geq 0\)
    \item \textbf{Bilateral:} if the constraint prevents displacements in one direction,
    it must prevent an equivalent displacement in the opposite direction, as well.
    These constraints generally represent bodies moving along a rail or a similar
    object. Mathematically, they are represented by
    equalities: \(f(\vv{q},\vd{q},t) = 0\)
  \end{itemize}
  \item Time-dependence:
  \begin{itemize}
    \item \textbf{Scleronomic:} the constraint does not depend explicity on time
    and can therefore be written as \(f(\vv{q},\vd{q})\).
      \item \textbf{Rheonomic:} the constraint depends explicitly on time.
  \end{itemize}
  \item Velocity-dependence:
  \begin{itemize}
    \item \textbf{Geometric:} the constraint does not depend explicity on velocity
    and can therefore be written as \(f(\vv{q},t)\).
    \item \textbf{Kinetic:} the constraint depends explicitly on velocity.
    \item \textbf{Integrable:} the constraint depends on velocity, but it is
    \emph{equivalent} to a geometric constraint; that is to say, there exists a
    geometric constraint that identifies the same subspace of allowed configurations.
    \item \textbf{Holonomic:} the constraint is geometric or integrable.
    \item \textbf{Anholonomic (nonholonomic):} the constraint is neither geometric,
    nor integrable. Unilateral constraints are generally considered anholonomic.
  \end{itemize}

\section{Udwadia-Kalaba equation [to be moved]}

Where \(\vv{Q}\) is the generalised force and \(\mathcal{Q}\) is the generalised constraint force.
The matrix \(M\) is symmetric and positive semi-definite.
%
\[M(\vv{q},t) \; \vdd{q} = \vv{Q}(\vv{q},\vd{q},t) + \mathcal{Q}(\vv{q},\vd{q},t)\]
\end{itemize}
