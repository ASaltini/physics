\chapter{General concepts}

\section{Generalised coordinates}

A set \(\{ q_i \}\) of \textbf{generalised coordinates} is a set of values that
identify the state of a physical system.
It can be useful to treat the set of generalised coordinates as a vector \(\vv{q}\).
It should be kept in mind that this is not a vector in the physical sense, so it
does not necessarily behave as expected under spatial transformations, it is merely
represented as such for the sake of mathematical convenience.

The least number of generalised coordinates that are necessary to uniquely
identify the state of a system is the number of \textbf{degrees of freedom} of
the system.
A physical system composed of \(N\) particles in 3 dimensions has, in the most
general case, \(3N\) degrees of freedom, as it is necessary to give the position
\(\vv{r}_s\) of each particle as a set of cartesian coordinates.
However, it is often the case that particles are not independent of each other;
consider the following examples:
\begin{itemize}
  \item The state of a rigid body can be described by simply giving the
  position of one of its points (3 vector components), and its orientation in space
  (3 angles), resulting in a total of only 6 degrees of freedom.
  \item The state of a thermodynamical system at equilibrium is described by its
  pressure and temperature, resulting in a total of only 2 degrees of freedom.
\end{itemize}

In a majority of cases, for each particle there exists an invertible map
\(\vv{r}_s = \vv{r}_s(\vv{q},t)\), that allows us to find the position of each
constituent of the system once we know the generalised coordinates.
This allows us to express \textbf{kinematic displacements} as
\begin{equation}
  \de\vv{r}_s = \dpd{\pderiv{\vv{r}_s}{\vv{q}}}{\de\vv{q}} + \pderiv{\vv{r}_s}{t} \, \de t
  \label{eq::kine-displacements}
\end{equation}
these are to be contrasted with \textbf{virtual displacements}
\begin{equation}
  \di\vv{r}_s = \dpd{\pderiv{\vv{r}_s}{\vv{q}}}{\de\vv{q}}
  \label{eq::virtual-displacements}
\end{equation}
The term \emph{virtual} comes from the fact that they happen at a fixed time
(they are obtained from real displacements by setting \(\de t = 0\)), but there
can obviously be no motion without passage of time, so the displacement is simply
a mathematical construct.

The time derivatives of generalised coordinates are called \textbf{generalised
velocities} (\(\vd{q}\)), and \textbf{generalised accelerations} (\(\vdd{q}\)).
As with generalised coordinates, it is possible to find the positions and
accelerations of each particle.
This is complicated by the fact that, while the cartesian coordinates only depend
on the generalised coordinates, it is not generally true that the cartesian velocities
and acceleration depend solely on their generalised equivalent, but rather
\(\vd{r}_s = \vd{r}_s(\vd{q},\vv{q},t)\) and \(\vdd{r}_s = \vdd{r}_s(\vdd{q},\vd{q},\vv{q},t)\).

We can compute the cartesian velocities \(\vd{r}_s\) by taking \eqref{eq::kine-displacements}.
\begin{equation}
  \vd{r}_s = \dpd{\pderiv{\vv{r}_s}{\vv{q}}}{\vd{q}} + \pderiv{\vv{r}_s}{t}
  \label{eq::cart-velocities}
\end{equation}
From this expression we observe that the cartesian velocities are linear in the
generalised velocities (while this is not generally true for the coordinates).

As with displacement, we can define a \textbf{kinematic differential of velocity}
\begin{equation}
  \de\vd{r}_s = \dpd{\pderiv{\vd{r}_s}{\vd{q}}}{\de\vd{q}} + \dpd{\pderiv{\vd{r}_s}{\vv{q}}}{\de\vv{q}} + \pderiv{\vd{r}_s}{t} \, \de t
  \label{eq::kine-velocities}
\end{equation}
and a \textbf{virtual or Jourdain differential of velocity}
\begin{equation}
  \di\vd{r}_s = \dpd{\pderiv{\vd{r}_s}{\vd{q}}}{\de\vd{q}}
  \label{eq::jourdain-velocities}
\end{equation}
that is obtained from the former by setting \(\de\vv{q} = 0\) and \(\de t = 0\).
By observing \eqref{eq::cart-velocities} we can compute the various terms in
\(\de\vd{r}_s\), that will be useful in computing the equation for the accelerations.
We take special note of the term that also appears in \(\di\vd{r}_s\).
\[\pderiv{\vd{r}_s}{\vd{q}} = \pderiv{\vv{r}_s}{\vv{q}} \qquad
\pderiv{\vd{r}_s}{\vv{q}} = \pderiv[2]{\vv{r}}{\vv{q}} \cdot \vd{q} + \pderiv{\vv{r}}{\vv{q},t} \qquad
\pderiv{\vd{r}_s}{t} = \pderiv{\vv{r}}{\vv{q},t} \cdot \vd{q} + \pderiv[2]{\vv{r}}{t}\]

We now take \eqref{eq::kine-velocities} and use it to compute the cartesian
accelerations \(\vdd{r}_s\).
\begin{align}
  \nonumber \vdd{r}_s &= \dpd{\pderiv{\vd{r}_s}{\vd{q}}}{\vdd{q}} +
          \dpd{\pderiv{\vd{r}_s}{\vv{q}}}{\vd{q}} + \pderiv{\vd{r}_s}{t} \\
            \vdd{r}_s &= \dpd{\pderiv{\vv{r}_s}{\vv{q}}}{\vdd{q}} +
          \dpd{\dpd{\pderiv[2]{\vv{r}_s}{\vv{q}}}{\vd{q}}}{\vd{q}} +
          2 \pderiv{\vv{r}_s}{\vv{q},t} \cdot \vd{q} + \pderiv[2]{\vv{r}_s}{t}
  \label{eq::cart-accelerations}
\end{align}
Here we see that the cartesian accelerations depend linearly on generalised
accelerations and may have both a quadratic and a linear term in the generalised
  velocities.


\section{Constraints}

A \textbf{constraint} is a relation between coordinates and velocities.
Using a set \(\vv{q}\) of generalised coordinates, a constrant can be represented
as an equality or inequality involving a \textbf{constraint function} \(f(\vv{q},\vd{q},t)\).

There is several ways to classify constraints.
\begin{itemize}
  \item Directionality:
  \begin{itemize}
    \item \textbf{Unilateral:} the constraint may prevent displacements in one
    direction, but not necessarily in the opposite direction. These constraints
    generally represent bodies that cannot be penetrated. Mathematically, they
    are represented by inequalities: \(f(\vv{q},\vd{q},t) \geq 0\)
    \item \textbf{Bilateral:} if the constraint prevents displacements in one direction,
    it must prevent an equivalent displacement in the opposite direction, as well.
    These constraints generally represent bodies moving along a rail or a similar
    object. Mathematically, they are represented by
    equalities: \(f(\vv{q},\vd{q},t) = 0\)
  \end{itemize}
  \item Time-dependence:
  \begin{itemize}
    \item \textbf{Scleronomic:} the constraint does not depend explicity on time
    and can therefore be written as \(f(\vv{q},\vd{q})\).
      \item \textbf{Rheonomic:} the constraint depends explicitly on time.
  \end{itemize}
  \item Velocity-dependence:
  \begin{itemize}
    \item \textbf{Geometric:} the constraint does not depend explicity on velocity
    and can therefore be written as \(f(\vv{q},t)\).
    \item \textbf{Kinetic:} the constraint depends explicitly on velocity.
    \item \textbf{Integrable:} the constraint is kinetic, but it is
    \emph{equivalent} to a geometric constraint. That is to say, there exists a
    geometric constraint that identifies the same subspace of allowed configurations.
    \item \textbf{Holonomic:} the constraint is geometric or integrable.
    \item \textbf{Anholonomic (nonholonomic):} the constraint is neither geometric,
    nor integrable. Unilateral constraints are generally considered anholonomic.
  \end{itemize}
\end{itemize}

A single bilateral holonomic constraint reduces the number of degrees of freedom
of a physical system by one. Bilateral holonomic constraints can always be expressed
as \(f(\vv{q},t) = 0\), we now pick a specific \(q_k\) and denote as \(\vv{\varrho}_k\)
the set of generalised coordinates \emph{except} \(q_k\), so that we can write
the constraint as \(f(q_k,\vv{\varrho}_k,t) = 0\).
Invoking Dini's implicit function theorem, there exists a function \(g(\vv{\varrho}_k,t)\)
such that \(f(g(\vv{\varrho}_k,t),\vv{\varrho}_k,t) = 0\).
Since the constraint must be satisfied, this forces us to choose
\(q_k = g(\vv{\varrho}_k,t)\).
The \(\vv{\varrho}_k\) are therefore the generalised coordinates for a new,
unconstrained system, with one fewer degree of freedom than the original.

\chapter{Alternative formulations [to be moved]}

\section{Udwadia-Kalaba equation}

Where \(\vv{Q}\) is the generalised force and \(\mathcal{Q}\) is the generalised
constraint force.
The matrix \(M\) is symmetric and positive semi-definite.
%
\[M(\vv{q},t) \; \vdd{q} = \vv{Q}(\vv{q},\vd{q},t) + \mathcal{Q}(\vv{q},\vd{q},t)\]

\section{Gauss's principle of least constraint}

The \textbf{Gauss's principle of least constraint} states that the function \(\mathcal{C}\)
defined as
%
\begin{equation}
  \mathcal{C} = \frac{1}{2} \Sum{m_s \nrm{\vdd{r}_s - \frac{\vv{F}_s}{m_s}}^2 }{s,1,N}
  \label{eq::least-constraint}
\end{equation}

must be minimised by the equations of motion.
Alternative ways to write \(\mathcal{C}\) include
%
\begin{subequations}
\begin{align}
  \label{eq::least-constraint1}
  \mathcal{C} &= \frac{1}{2} \Sum{m_s \dpd{\of{\vdd{r}_s - \frac{\vv{F}_s}{m_s}}}{\of{\vdd{r}_s - \frac{\vv{F}_s}{m_s}}}}{s,1,N} \\
  \label{eq::least-constraint2}
  \mathcal{C} &= \frac{1}{2} \Sum{\of{m_s \dpd{\vdd{r}_s}{\vdd{r}_s} - 2 \dpd{\vv{F}_s}{\vdd{r}_s} + \frac{\dpd{\vv{F}_s}{\vv{F}_s}}{m_s}}}{s,1,N} \\
  \label{eq::least-constraint3}
  \mathcal{C} &= \frac{1}{2} \Sum{\of{m_s \nrm{\vdd{r}_s}^2 - 2 \dpd{\vv{F}_s}{\vdd{r}_s} + \frac{\nrm{\vv{F}_s}^2}{m_s}}}{s,1,N}
\end{align}
\end{subequations}

\section{Jourdain's principle}

D'Alembert principle only applies to holonomic systems, a more general principle
that can also be applied to anholonomic system can be obtained by working with the
virtual velocity differentials from \ref{eq::jourdain-velocities}.
As it was the case with d'Alembert's principle, if the boundaries are smooth, then
the constraint forces must be normal to the constraint surface.
Much like the virtual displacements, the virtual velocity differential must also
lie on the constraint surface, so that \(\dpd{\vv{\Phi}_s}{\di\vd{r}_s} = 0\).
By writing \(\vv{\Phi}_s\) out as \(m_s \vdd{r}_s - \vv{F}_s\) we obtain the
statement of \textbf{Jourdain's principle}.
%
\[\Sum{\dpd{\of{m_s \vdd{r}_s - \vv{F}_s}}{\di\vd{r}_s}}{s,1,N} = 0\]

\section{Gauss-Gibbs principle}

Another principle can be obtained by working with virtual changes in acceleration,
this is known as the \textbf{Gauss-Gibbs principle}.
%
\[\Sum{\dpd{\of{m_s \vdd{r}_s - \vv{F}_s}}{\di\vdd{r}_s}}{s,1,N} = 0\]

\section{Mangeron-Deleanu principle}

The most general case is known as \textbf{Mangeron-Deleanu principle} and is
stated in terms of a virtual variation of the \(n\)-th time derivative of the
cartesian positions.
%
\[\Sum{\dpd{\of{m_s \vdd{r}_s - \vv{F}_s}}{\di\vv{r}_s^{(n)}}}{s,1,N} = 0\]

\section{Appell's equation}

We take the principle of least constraint (\eqref{eq::least-constraint3}) and we
define the \textbf{Appellian} \(\mathcal{A}\) as
%
\[\mathcal{A} = \frac{1}{2} \Sum{m_s \nrm{\vdd{r}_s}^2}{s,1,N}\]
%
which we recognise as the first term in \(\mathcal{C}\).
\[\mathcal{C} = \mathcal{A} + \frac{1}{2} \Sum{\of{\frac{\nrm{\vv{F}_s}^2}{m_s} - 2 \dpd{\vv{F}_s}{\vdd{r}_s}}}{s,1,N}\]
%
Since \(\mathcal{C}\) must be minimal, its derivative with regards to the
\(\vdd{q}\) must be zero.
Assuming the forces are adequately well-behaved (it is sufficient that they do
not depend on the accelerations), their derivative with regards to the \(\vdd{q}\)
will also be zero.
We also recall \eqref{eq::cart-accelerations} to find the derivative of the
cartesian accelerations.
\[0 = \pderiv{\mathcal{C}}{\vdd{q}} = \pderiv{\mathcal{A}}{\vdd{q}} - \Sum{\dpd{\pderiv{\vdd{r}_s}{\vdd{q}}}{\vv{F}_s}}{s,1,N}
  = \pderiv{\mathcal{A}}{\vdd{q}} - \Sum{\dpd{\pderiv{\vv{r}_s}{\vv{q}}}{\vv{F}_s}}{s,1,N}\]
The second term gives the generalised forces \(\vv{Q}\), we now have \textbf{Appell's equation}.
%
\[\vv{Q} = \pderiv{\mathcal{A}}{\vdd{q}}\]
